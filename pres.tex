\documentclass{beamer}
\usetheme{Szeged}
\usecolortheme{beaver}

\newenvironment{itemize*}%
  {\begin{itemize}%
    \setlength{\itemsep}{0.5\baselineskip}}%
  {\end{itemize}}

\begin{document}
\title{Your CS Career 2014}
\author{Greg Hill\\ Bryan Cuccioli\\ Jeremy Feinstein\\ Jisun Jung}
\date{May 6, 2014}

\frame{\titlepage}

\section{Industry}
\begin{frame}[plain,c]
  \begin{center}{\huge Industry}\end{center}
\end{frame}

\frame{\frametitle{Big Company}
% TODO - Bryan?
}

\frame{\frametitle{Startups}
\begin{itemize*}
	\item Often looking for generalists
	\begin{itemize*}
		\item R\&D, back-end, front-end, mobile, \boxed{\text{all of the above}}
	\end{itemize*}
	\item Often looking for very polished engineers
	\begin{itemize*}
		\item They won't have much time to train you
	\end{itemize*}
	\item Usually a fast, hectic environment
	\item If you're good at what you do, then you'll actually have more job
    security than in, say, the financial industry. And in the case that the
    company does run out of money, you'll have no trouble finding your next gig.
	\item Be prepared to take a salary cut for some significant equity and the
    chance to learn
\end{itemize*}
}

\frame{\frametitle{Timeline}
\begin{itemize*}
  \item Find job opportunities and apply (fall semester)
  \item Interview (a few weeks later)
  \item Choose from among your offers
\end{itemize*}
}

\frame{\frametitle{Applying}
\begin{itemize*}
  \item Go to career fairs (happens \emph{early} in the year) \& use CCNet
%  \begin{itemize*}
%    \item Generally the best way to find a job
%  \end{itemize*}
  \item Join the ACSU mailing list and Facebook group
  \item Research companies on the internet, don't be afraid to email ones that
    you're interested in even if they don't have job postings
  \item Information sessions
  \item Past experience most important % and GPA most important (I disagree with GPA, I was never once asked about it - Jeremy)
  \begin{itemize*}
    \item Side projects are awesome and can generate inbound recruitment
    \item Mention relevant courses
  \end{itemize*}
\end{itemize*}
}

\frame{\frametitle{Interviewing}
\begin{itemize*}
  \item Review data structures, algorithms and techniques
  \item Practice, practice, practice
  \item Trade questions with friends, make it a game
  \item Fail a few interviews first
  \begin{itemize*}
    \item Interviewing well takes practice and experience
  \end{itemize*}
\end{itemize*}
}

\frame{\frametitle{Choosing Among Offers}
\begin{itemize*}
  \item Location
  \item Salary
  \item Type of position and technologies used
  \item Company Culture
  \item Startup vs. large company
\end{itemize*}
}

\section{Graduate School}
\frame{\frametitle{Graduate School}
The common options:
\begin{itemize*}
  \item M.Eng in CS at Cornell
  \item M.S. in CS at another school
  \item Ph.D. in CS
\end{itemize*}
}

\frame{\frametitle{M.Eng. in CS at Cornell}
This is primarily a terminal degree targeted at industry.

What it entails:
\begin{itemize*}
  \item An additional year (or semester) at Cornell
  \item A large project or independent research
\end{itemize*}

Why you should do it:
\begin{itemize*}
  \item You want skills from extra advanced courses
  \item You want to improve your job prospects
  \item You want a higher starting salary in the industry.
\end{itemize*}
}

\frame{\frametitle{M.S. at Another University}
These are typically two year programs that are likely targeted at industry,
but can also lead to a Ph.D.

What it entails:
\begin{itemize*}
  \item Usually significant coursework component
  \item Also could be research-focused MS ending in thesis
\end{itemize*}

Why you should do it:
\begin{itemize*}
  \item You want to significantly improve the breadth and depth of your CS
    knowledge through more classes and/or research.
  \item You're thinking about doing a Ph.D. but didn't do much/any undergraduate
    research.
\end{itemize*}
}

\frame{\frametitle{Ph.D.}
Best described in one word: \textbf{research}

What it entails:
\begin{itemize*}
  \item Likely 5-6 years of primarily working on research
  \item Coursework component typically quite small
\end{itemize*}

Why you should do it:
\begin{itemize*}
  \item You really love research and are committed to pursuing a career in
    academic research or an industry research lab.
\end{itemize*}
}

\frame{\frametitle{Preparation for a Ph.D. Program}
As a sophomore and junior, just do research!

\begin{itemize*}
  \item Finding a professor
    \begin{itemize*}
      \item Course you enjoyed
      \item Check webpages, send emails
      \item Don't be discouraged if you don't get replies
    \end{itemize*}
  \item Make sure you're making meaningful contributions
  \item Work towards a publication
\end{itemize*}
}

\frame{\frametitle{Applying to a Ph.D. Program}
Application requirements:

\begin{itemize*}
  \item Three letters of recommendation
  \item Statement of purpose (\textbf{research statement})
  \item Honors/publications (optional)
  \item Transcript
  \item GRE scores (general only, don't take CS subject)
\end{itemize*}

Also, apply to fellowships:

\begin{itemize*}
  \item Looks good
  \item Possible to be admitted without funding
\end{itemize*}
}

\section{Miscellaneous}
\begin{frame}[plain,c]
  \begin{center}{\huge Miscellaneous}\end{center}
\end{frame}

\frame{\frametitle{Useful Things at Cornell}
\begin{itemize*}
  \item Join a course staff
  \item Project teams
  \item 3 Day Startup
  \item Hackathons
  \item POPSHOP
  \item The (extended) CS community (ACSU, ISSA, CUxD)
  \item Open Source Projects (Firefox, GNU, etc.)
\end{itemize*}
}

\section{Questions}
\begin{frame}[plain,c]
  \begin{center}{\huge Questions?}\end{center}
\end{frame}

\end{document}
