\documentclass{beamer}
\usetheme{Szeged}
\usecolortheme{beaver}

\newenvironment{itemize*}%
  {\begin{itemize}%
    \setlength{\itemsep}{0.5\baselineskip}}%
  {\end{itemize}}

\begin{document}
\title{Your CS Career 2014}
\author{Bryan Cuccioli\\Jeremy Feinstein\\Greg Hill\\Jisun Jung}
\date{May 6, 2014}

\frame{\titlepage}

\section{Undergrad at Cornell}
\begin{frame}[plain,c]
  \begin{center}{\huge Coursework}\end{center}
\end{frame}

\frame{\frametitle{Coursework at Cornell}
\begin{itemize*}
  \item Not all courses are created equal!
  \item Put the most work into the CS Core: 3110, 3410, 4410, 4820!
  \item While other classes might focus on topics you're more interested in,
    these are your core competencies.
\end{itemize*}
}

\frame{\frametitle{Choosing Courses}
\begin{itemize*}
  \item The first digit is not always correlated with difficulty
  \item Don't be afraid to explore subjects you aren't familiar with
  \begin{itemize*}
    \item Common: I thought I would like X but I didn't, I never tried Y but
      now I really like it.
  \end{itemize*}
  \item Don't lock yourself in to just one vector!
\end{itemize*}
}

\frame{\frametitle{Courses of Interest}
\begin{itemize*}
  \item \textbf{Theory}: 4820, 4810, 6820, 6810, 6850
  \item \textbf{PL}: 4110, 6110, 6118, 4120/4121
  \item \textbf{Systems}: 4410, 3410, 6410, 4411 highly recommended
  \item \textbf{AI}: 4700/4701, 4740, 4780, 4670
  \item \textbf{Math}: analysis, algebra, combinatorics, logic, topology
\end{itemize*}
}

\frame{\frametitle{Useful Things at Cornell}
\begin{itemize*}
  \item Join a course staff
  \item Project teams
  \item Association of Computer Science Undergraduates
  \item Women in Computing at Cornell
  \item BigRed//Hacks
  \item The (extended) CS community (ISSA, CUxD, POPSHOP, Cornell Opensource)
  \item Open Source Projects (Firefox, GNU, etc.)
  \\
  \item However, this isn't highschool: don't spread self too thin
\end{itemize*}
}


\section{Industry}
\begin{frame}[plain,c]
  \begin{center}{\huge Industry}\end{center}
\end{frame}

\frame{\frametitle{Big Company}
\begin{itemize*}
  \item Experiences can be mixed
  \item Generally hire students with a breadth of skills
  \item Even though you will probably work on one type of thing (e.g. iOS,
    compilers, systems, front end)
  \item Advantage: Lots of resources, perks, housing, other interns,
    established intern and orientation programs
  \item Disadvantage: You can fall through the cracks
\end{itemize*}
}

\frame{\frametitle{Startups}
\begin{itemize*}
	\item Often looking for generalists
	\begin{itemize*}
		\item R\&D, back-end, front-end, mobile, \boxed{\text{all of the above}}
	\end{itemize*}
	\item Often looking for very polished engineers
	\begin{itemize*}
		\item They won't have much time to train you
	\end{itemize*}
	\item Usually a fast, hectic environment
	\item If you're good at what you do, then you'll actually have more job
    security than in, say, the financial industry. And in the case that the
    company does run out of money, you'll have no trouble finding your next gig.
	\item Be prepared to take a salary cut for some significant equity and the
    chance to learn
\end{itemize*}
}

\frame{\frametitle{Timeline}
\begin{itemize*}
  \item Find job opportunities and apply (fall semester)
  \item Interview (a few weeks later)
  \item Choose from among your offers
\end{itemize*}
}

\frame{\frametitle{Applying}
\begin{itemize*}
  \item Go to career fairs (happens \emph{early} in the year) \& use CCNet
%  \begin{itemize*}
%    \item Generally the best way to find a job
%  \end{itemize*}
  \item Join the ACSU mailing list and Facebook group
  \item Research companies on the internet, don't be afraid to email ones that
    you're interested in even if they don't have job postings
  \item Information sessions
  \item Past experience most important
  \item High GPA can help you get to the interview stage
  \begin{itemize*}
    \item Side projects are awesome and can generate inbound recruitment
    \item Mention relevant courses
  \end{itemize*}
\end{itemize*}
}

\frame{\frametitle{Interviewing}
\begin{itemize*}
  \item Review data structures, algorithms and techniques
  \item Practice, practice, practice
  \item Trade questions with friends, make it a game
  \item Fail a few interviews first
  \begin{itemize*}
    \item Interviewing well takes practice and experience
  \end{itemize*}
\end{itemize*}
}

\frame{\frametitle{Choosing Among Offers}
\begin{itemize*}
  \item Location
  \item Salary, Stock, Signing Bonuses
  \item Type of position and technologies used
  \item Company Size and Culture
  \item Negotiate!
  \begin{itemize*}
    \item Chance at larger salary with minimal effort (\$5-\$20k more in some cases)
    \item Phrase nicely in a way they can decline without rescinding offer, along the lines of: "I'm really interested in working at your company but I also have offer $X$ from company $Y$... it would make my decision easier if you could offer me $\$Z$ more"
  \end{itemize*}
\end{itemize*}
}

\section{Graduate School}
\begin{frame}[plain,c]
  \begin{center}{\huge Graduate School}\end{center}
\end{frame}

\frame{\frametitle{Graduate School}
The common options:
\begin{itemize*}
  \item M.Eng in CS at Cornell
  \item M.S. in CS at another school
  \item Ph.D. in CS
\end{itemize*}
}

\frame{\frametitle{M.Eng. in CS at Cornell}
This is primarily a terminal degree targeted at industry.

What it entails:
\begin{itemize*}
  \item An additional year (or semester) at Cornell
  \item A large project or independent research
\end{itemize*}

Why you should do it:
\begin{itemize*}
  \item You want skills from extra advanced courses
  \item You want to improve your job prospects
  \item You want a higher starting salary in the industry.
\end{itemize*}
}

\frame{\frametitle{M.S. at Another University}
These are typically two year programs that are likely targeted at industry,
but can also lead to a Ph.D.

What it entails:
\begin{itemize*}
  \item Usually significant coursework component
  \item Also could be research-focused MS ending in thesis
\end{itemize*}

Why you should do it:
\begin{itemize*}
  \item You want to significantly improve the breadth and depth of your CS
    knowledge through more classes and/or research.
  \item You're thinking about doing a Ph.D. but didn't do much/any undergraduate
    research.
\end{itemize*}
}

\frame{\frametitle{Ph.D.}
Best described in one word: \textbf{research}

What it entails:
\begin{itemize*}
  \item Likely 5-6 years of primarily working on research
  \item Coursework component typically quite small
\end{itemize*}

Why you should do it:
\begin{itemize*}
  \item You really love research and/or are interested in pursuing a career in
    academia or an industry research lab. (But can also help for normal software development)
\end{itemize*}
}

\frame{\frametitle{Preparation for a Ph.D. Program}
As a sophomore or junior, just do research!

\begin{itemize*}
  \item Finding a professor
    \begin{itemize*}
      \item Course you enjoyed
      \item Check webpages, send emails
      \item Don't be discouraged if you don't get replies
    \end{itemize*}
  \item Make sure you're making meaningful contributions
  \item Work towards a publication
  \item Let professor know from beginning you are interested in applying to PhD (they will work more to ensure two above)
\end{itemize*}
}

\frame{\frametitle{Applying to a Ph.D. Program}
Application requirements (roughly by importance):

\begin{itemize*}
  \item Three letters of recommendation
  \item Statement of purpose (\textbf{research statement})
  \item Honors/publications (optional)
  \item Transcript
  \item GRE scores (general only, don't take CS subject)
\end{itemize*}

Also, apply to fellowships:

\begin{itemize*}
  \item Looks good
  \item Possible to be admitted without funding
\end{itemize*}
}

\section{Questions}
\begin{frame}[plain,c]
  \begin{center}{\huge Questions?}\end{center}
\end{frame}

\end{document}
